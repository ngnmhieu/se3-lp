\documentclass[]{article}
\usepackage{lmodern}
\usepackage{amssymb,amsmath}
\usepackage{ifxetex,ifluatex}
\usepackage{fixltx2e} % provides \textsubscript
\ifnum 0\ifxetex 1\fi\ifluatex 1\fi=0 % if pdftex
  \usepackage[T1]{fontenc}
  \usepackage[utf8]{inputenc}
\else % if luatex or xelatex
  \ifxetex
    \usepackage{mathspec}
  \else
    \usepackage{fontspec}
  \fi
  \defaultfontfeatures{Ligatures=TeX,Scale=MatchLowercase}
\fi
% use upquote if available, for straight quotes in verbatim environments
\IfFileExists{upquote.sty}{\usepackage{upquote}}{}
% use microtype if available
\IfFileExists{microtype.sty}{%
\usepackage{microtype}
\UseMicrotypeSet[protrusion]{basicmath} % disable protrusion for tt fonts
}{}
\usepackage[unicode=true]{hyperref}
\hypersetup{
            pdfborder={0 0 0},
            breaklinks=true}
\urlstyle{same}  % don't use monospace font for urls
\IfFileExists{parskip.sty}{%
\usepackage{parskip}
}{% else
\setlength{\parindent}{0pt}
\setlength{\parskip}{6pt plus 2pt minus 1pt}
}
\setlength{\emergencystretch}{3em}  % prevent overfull lines
\providecommand{\tightlist}{%
  \setlength{\itemsep}{0pt}\setlength{\parskip}{0pt}}
\setcounter{secnumdepth}{0}
% Redefines (sub)paragraphs to behave more like sections
\ifx\paragraph\undefined\else
\let\oldparagraph\paragraph
\renewcommand{\paragraph}[1]{\oldparagraph{#1}\mbox{}}
\fi
\ifx\subparagraph\undefined\else
\let\oldsubparagraph\subparagraph
\renewcommand{\subparagraph}[1]{\oldsubparagraph{#1}\mbox{}}
\fi

% set default figure placement to htbp
\makeatletter
\def\fps@figure{htbp}
\makeatother


\date{}

\begin{document}

\section{SEIII - Logikprogrammierung}\label{header-n93}

\subsection{Übungsblatt 04}\label{header-n95}

Nico Hahn 6990715 Hieu Nguyen 6632126

\subsubsection{Aufgabe 1}\label{header-n99}

\begin{itemize}
\item
  A ist Nachbarland von B - symmetrisch und vllt. reflexiv: B ist auch
  Nachbarland von A wenn A Nachbarland von B ist. Man kann auch sagen
  dass die Relation reflexiv ist - A ist auch ein Nachbarland von A.
\end{itemize}

\begin{itemize}
\item
  A und B sind (nach deutschem Recht) miteinander verheiratet -
  symmetrisch: A kann nicht sich selbst heiraten - also nicht reflexiv.
  A und B sind verheiratet bedeutet dann sind B und A auch.
\item
  A und B sind Geschwister - symmetrisch,s ggf. reflexiv:
  Geschwister-Relation zwischen A und B ist klar symmetrisch. Sie ist
  dann reflexiv, wenn die Definition von Geschwister ist, dass sie die
  gleichen Eltern haben. Dann sind A und C auch Geschwister falls B und
  C Geschwister sind. 
\item
  A ist größer oder gleich B - reflexiv und transitiv: A ist
  selbsverständlich größer oder gleich A (reflexiv). Falls B größer oder
  gleich C ist, dann ist A größer oder gleich C.
\item
  A und B haben ein gemeinsames Hobby - symmetrisch und reflexiv: A und
  B haben ein gemeinsames Hobby dann B und A haben auch ein gemeinsames
  Hobby. A hat selbstverständlich mit A ein gemeinsames Hobby. Falls B
  und C ein gemeinsames Hobby bedeutet es nicht, dass dieses Hobby
  gleich das Hobby zwischen A und B. Daher muss A und C auch nicht ein
  gemeinsames Hobby haben.
\item
  A und B sind Häuser in der gleichen Straße - symmetrisch, reflexiv und
  transitiv: A ist mit A zusammen in der gleichen Straße - das ist klar
  (reflexiv). A und B sind in der gleichen Straße dann B und A auch
  (symmetrisch). A und B sind in der gleichen Straße und B und C sind in
  der gleichen Straße; da es nur eine Straße sein kann, sind A und C
  auch Häuser in der gleichen Straße (transitiv).
\end{itemize}

\subsubsection{Aufgabe 2}\label{header-n129}

\begin{verbatim}
?- consult('dateiverzeichnis.pl').

% 2.1 Zugriffpfad zwischen zwei Verzeichnissen
% zugriffpfad(?DirId1, ?DirId2)
con_sym(DId1, DId2) :- directory(DId1,_,DId2,_,_).
con_sym(DId1, DId2) :- directory(DId2,_,DId1,_,_).
zugriffpfad(DId1, DId2) :- con_sym(DId1,DId2).
zugriffpfad(DId1, DId2) :- con_sym(DId1,DId3), zugriffpfad(DId3, DId2).

% 2.2 Ueberpreuft, ob eine Datei vom Wurzelverzeichnis aus nicht mehr erreichbar ist
% nicht_zugreifbar(?FId)
wurzel_dir(DId) :- directory(DId,_,1,_,_).
wurzel_dir(DId) :- directory(DId,_,PId,_,_), wurzel_dir(PId).
wurzel_datei(FId) :- file(FId,DId,_,_,_,_), wurzel_dir(DId).
nicht_zugreifbar(FId) :- \+ wurzel_datei(FId).

% 2.3 Direkte und indirekte Unterverzeichnisse ermitteln
% uv(+PId,-UnterDId)
uv(PId,UnterDId) :- directory(UnterDId,_,PId,_,_).
uv(PId,UnterDId) :- directory(UnterDId,_,UnterDId2,_,_),
                    uv(PId,UnterDId2).

% 2.4 Gesamtgröße aller Datein
% sumsize(+DId, -Size)
filesizes(DId, Size) :- file(_,DId,_,Size,_,_);
                        (uv(DId,UnterDId),file(_,UnterDId,_,Size,_,_)).
sumsize(DId, Size) :- findall(FSize,filesizes(DId,FSize),SizeList),
                      sumlist(SizeList, Size).
\end{verbatim}

\subsubsection{Aufgabe 3}\label{header-n131}

\begin{verbatim}
?- consult('skigebiet.pl').
% Aufgabe 3.1
is_highest(P) :- strecke(_,P,_,_,_), \+ strecke(_,_,P,_,_).
% Hoechste Punkte treten niemals als ZeilPunkt auf 
% -> Hoechste Punkte sind bgrootmoos, bgpanorama und bgzirben
is_lowest(P) :- strecke(_,_,P,_,_), \+ strecke(_,P,_,_,_).
% Niedrigste Punkte treten niemals als StartPunkt auf
% -> Niedrigste Punkte sind tlzollberg und tlgondel

% Aufgabe 3.2
% ist_erreichbar(?Start,?Ziel)
ist_erreichbar_piste(Start,Ziel,PNr) :-
  strecke(_,Start,Ziel,PNr,_).
ist_erreichbar_piste(Start,Ziel,PNr) :-
  strecke(_,Start,ZielX,PNr,_),
  ist_erreichbar_piste(ZielX,Ziel,PNr).
ist_erreichbar(Start,Ziel) :- ist_erreichbar_piste(Start,Ziel,_).

% Es ist nicht terminierungssicher. Wäre ein Strecke falsch definiert,
% sodass ein Kreis entsteht, % dann haben wir eine endlose Schleife.
% Eigenschaft: transitiv - falls es einen Pfad von A zu B und von B zu C
% dann gibt es auch einen Pfad von A zu C.

% Aufgabe 3.3
% keine_verbindung(?OrtA,?OrtB) 
% pit_con_sym prueft ob zwei Orte mit einander verbunden sind
pit_con_sym(OrtA, OrtB) :- ist_erreichbar(OrtA, OrtB).
pit_con_sym(OrtA, OrtB) :- ist_erreichbar(OrtB, OrtA).
keine_verbindung(OrtA, OrtB) :- 
  (strecke(_,OrtA,_,_,_); strecke(_,_,OrtA,_,_)),
  (strecke(_,OrtB,_,_,_); strecke(_,_,OrtB,_,_)),
  OrtA \= OrtB,
  \+ pit_con_sym(OrtA, OrtB).

% Aufgabe 3.4
% treffpunkt(+OrtA, +OrtB, -Treffpunkt)
treffpunkt(OrtA, OrtB, X) :- OrtA = OrtB, X = OrtA.
treffpunkt(OrtA, OrtB, X) :- ist_erreichbar(OrtA, OrtB), X = OrtB.
treffpunkt(OrtA, OrtB, X) :- ist_erreichbar(OrtB, OrtA), X = OrtA.
treffpunkt(OrtA, OrtB, X) :- ist_erreichbar(OrtA, X),
                             ist_erreichbar(OrtB, X).

% Aufgabe 3.5
:- dynamic(gesperrt/1).
% ist_erreichbar_sperre(?Start,?Ziel)
hilf_ist_erreichbar_sperre(StrNr,Start,Ziel,PNr) :-
  strecke(StrNr,Start,Ziel,PNr,_),
  \+ gesperrt(StrNr).
hilf_ist_erreichbar_sperre(StrNr,Start,Ziel,PNr) :-
  strecke(StrNr,Start,ZielMittel,PNr,_),
  \+ gesperrt(StrNr),
  hilf_ist_erreichbar_sperre(_,ZielMittel,Ziel,PNr).
ist_erreichbar_sperre(Start,Ziel) :- hilf_ist_erreichbar_sperre(_,Start,Ziel,_).
\end{verbatim}

\end{document}
