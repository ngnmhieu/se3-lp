\documentclass[]{article}
\usepackage{lmodern}
\usepackage{amssymb,amsmath}
\usepackage{ifxetex,ifluatex}
\usepackage{fixltx2e} % provides \textsubscript
\ifnum 0\ifxetex 1\fi\ifluatex 1\fi=0 % if pdftex
  \usepackage[T1]{fontenc}
  \usepackage[utf8]{inputenc}
\else % if luatex or xelatex
  \ifxetex
    \usepackage{mathspec}
  \else
    \usepackage{fontspec}
  \fi
  \defaultfontfeatures{Ligatures=TeX,Scale=MatchLowercase}
\fi
% use upquote if available, for straight quotes in verbatim environments
\IfFileExists{upquote.sty}{\usepackage{upquote}}{}
% use microtype if available
\IfFileExists{microtype.sty}{%
\usepackage{microtype}
\UseMicrotypeSet[protrusion]{basicmath} % disable protrusion for tt fonts
}{}
\usepackage[unicode=true]{hyperref}
\hypersetup{
            pdfborder={0 0 0},
            breaklinks=true}
\urlstyle{same}  % don't use monospace font for urls
\IfFileExists{parskip.sty}{%
\usepackage{parskip}
}{% else
\setlength{\parindent}{0pt}
\setlength{\parskip}{6pt plus 2pt minus 1pt}
}
\setlength{\emergencystretch}{3em}  % prevent overfull lines
\providecommand{\tightlist}{%
  \setlength{\itemsep}{0pt}\setlength{\parskip}{0pt}}
\setcounter{secnumdepth}{0}
% Redefines (sub)paragraphs to behave more like sections
\ifx\paragraph\undefined\else
\let\oldparagraph\paragraph
\renewcommand{\paragraph}[1]{\oldparagraph{#1}\mbox{}}
\fi
\ifx\subparagraph\undefined\else
\let\oldsubparagraph\subparagraph
\renewcommand{\subparagraph}[1]{\oldsubparagraph{#1}\mbox{}}
\fi

% set default figure placement to htbp
\makeatletter
\def\fps@figure{htbp}
\makeatother


\date{}

\begin{document}

\section{SEIII - Logikprogrammierung}\label{header-n92}

\subsection{Übungsblatt 02-03}\label{header-n94}

Nico Hahn 6990715 Hieu Nguyen 6632126

\subsubsection{Aufgabe 1}\label{header-n98}

\begin{enumerate}
\def\labelenumi{\arabic{enumi}.}
\item
  Ein Prädikat umfasst eine Folge von Wörtern mit klar definierten
  Leerstellen welches zu einer wahren oder falschen Aussage ausgewertet
  werden kann. Eine Klausel bezeichnet eine Boolsche Funktion, welche
  nur aus distinktiven Verknüpfungen von Literalen entsteht. 
\item
  Im Hinblick auf die Logikprogrammierung ist die Suche eine Vorgang,
  welcher eine im System eingebundene Datenstruktur sequenziell
  durchläuft und angefragt Ergebnisse ausgibt. Eine Variable macht es
  möglich, nach dem sie mit dem default Wert instanziiert wurde,
  Ergebnisse in ihr abzuspeichern und mit diesen Werten weiter arbeiten
  zu können. Eine Instanziierung wie schon aus dem vorherigen Satz
  hervorgehend, ist eine Zuweisung eines Wertes an eine Variable. 
\end{enumerate}

\subsubsection{Aufgabe 2}\label{header-n106}

\begin{verbatim}
% 2.1 Welche Haeuser stehen in der Bahnhofsstrasse?

bahnHofHaeuser(ObjNr, HausNr) :-
   obj(ObjNr,_,bahnhofsstr,HausNr,_).

% Ausgabe: 
%  ObjNr = 2,
%  HausNr = 27,
%  ObjNr = 3,
%  HausNr = 29,
%  ObjNr = 4,
%  HausNr = 28,
%  ObjNr = 5,
%  HausNr = 30,
%  ObjNr = 6,
%  HausNr = 26,

% 2.2 Welche Haeuser wurden vor 1950 gebaut?

altBau(ObjNr,StrName,HausNr,BauJahr) :- 
   obj(ObjNr,_,StrName,HausNr,BauJahr), BauJahr < 1950.
   
% Ausgabe: 
%  ObjNr = 2,
%  StrName = bahnhofsstr,
%  HausNr = 27,
%  BauJahr = 1943 ;
%  ObjNr = 5,
%  StrName = bahnhofsstr,
%  HausNr = 30,
%  BauJahr = 1901 ;
%  false.

% 2.3 Wer besitzt Haeuser, die mehr als 300.000 Euro wert sind?

teuerHausBesitzer(Besitzer, Preis) :-
   bew(_,_,_,Besitzer,Preis,_), Preis > 300000.
   
% Ausgabe:
%  Besitzer = mueller,
%  Preis = 315000 ;
%  Besitzer = piepenbrink,
%  Preis = 1500000.

% 2.4 Welche Haeuser wurden mit Gewinn weiterverkauft

gewinnVerkauft(ObjNr, StrName, HausNr) :- 
   obj(ObjNr,_,StrName,HausNr,_),
   bew(_,ObjNr,_,Kaeufer,KaufPreis,_), 
   bew(_,ObjNr,Kaeufer,_,VerkaufPreis,_),
   VerkaufPreis > KaufPreis.
   
% Ausgabe:
%  ObjNr = 3,
%  StrName = bahnhofsstr,
%  HausNr = 29 ;
%  ObjNr = 3,
%  StrName = bahnhofsstr,
%  HausNr = 29 ;
%  false.

% 2.5 Welche Haeuser haben schon mehrfach den Besitzer gewechselt

mehrGekauft(ObjNr, StrName, HausNr) :- 
   bew(_,ObjNr,_,Besitzer1,_,_),
   bew(_,ObjNr,_,Besitzer2,_,_),Besitzer1 \= Besitzer2,
   obj(ObjNr,_, StrName, HausNr, _).
   
% Ausgabe:
%  ObjNr = 3,
%  StrName = bahnhofsstr ;
%  ObjNr = 3,
%  StrName = bahnhofsstr ;
%  false.

% 2.6
% TODO
\end{verbatim}

\subsubsection{Aufgabe 3}\label{header-n108}

\begin{verbatim}
% 3.1 Umrechnen zwischen Dateinamen und Dateischluessel
% fNameToFId(?FId,?FName)
fNameToFId(FId,FName) :- file(FId,_,FName,_,_,_).

% 3.2 Umrechnen zwischen Verzeichnisnamen und Verzeichnisschluessel
% dNameToDId(?DId,?DName)
dNameToDId(DId,DName) :- directory(DId,DName,_,_,_).

% 3.3 Name und Schluessel des zugehörigen Verzeichnises der Datei
% fileToDir(+FileName, -DirId, -DirName)
fileToDir(FileName, DirId, DirName) :- file(_,DirId,FileName,_,_,_),
                                       directory(DirId,DirName,_,_,_).

% 3.4 Name und Schluessel des übergeordnetes Verzeichnises
% parentDir(+DirName, -ParentId, -ParentName)
parentDir(DirName, ParentId, ParentName) :-
    directory(_,DirName,ParentId,_,_),
																						directory(ParentId,ParentName,_,_,_).
\end{verbatim}

\subsubsection{Aufgabe 4}\label{header-n110}

\begin{verbatim}
% 4.1 Dateinamen
% listFile(+DirId, -FileList)
listFile(DirId, FileList) :- 
  findall(FileName,file(_,DirId,FileName,_,_,_),FileList).

% 4.2 Namen aller Unterverzeichnisse
% listSubdir(+ParentDirId, -DirList)
listSubdir(ParentDirId, DirList) :- 
  findall(SubDir,directory(_,SubDir,ParentDirId,_,_),DirList).

% 4.3 Anzahl aller Datein in einem bestimmten Verzeichnis
% countFiles(+DirId, -Count) 
countFiles(DirId, Count) :- listFile(DirId, FileList),
                            length(FileList,Count).
\end{verbatim}

\end{document}
